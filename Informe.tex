%%%%%%%%%%%%%%%%%%%%%%%%%%%%%%%%%%%%%%%%%
% University/School Laboratory Report
% LaTeX Template
% Version 3.1 (25/3/14)
%
% This template has been downloaded from:
% http://www.LaTeXTemplates.com
%
% Original author:
% Linux and Unix Users Group at Virginia Tech Wiki 
% (https://vtluug.org/wiki/Example_LaTeX_chem_lab_report)
%
% License:
% CC BY-NC-SA 3.0 (http://creativecommons.org/licenses/by-nc-sa/3.0/)
%
%%%%%%%%%%%%%%%%%%%%%%%%%%%%%%%%%%%%%%%%%

%----------------------------------------------------------------------------------------
%	PACKAGES AND DOCUMENT CONFIGURATIONS
%----------------------------------------------------------------------------------------

\documentclass{article}

\usepackage[version=3]{mhchem} % Package for chemical equation typesetting
\usepackage{siunitx} % Provides the \SI{}{} and \si{} command for typesetting SI units
\usepackage{graphicx} % Required for the inclusion of images
\usepackage{natbib} % Required to change bibliography style to APA
\usepackage{amsmath} % Required for some math elements
\usepackage{makeidx} 

\setlength\parindent{0pt} % Removes all indentation from paragraphs

\renewcommand{\labelenumi}{\alph{enumi}.} % Make numbering in the enumerate environment by letter rather than number (e.g. section 6)

%\usepackage{times} % Uncomment to use the Times New Roman font

%----------------------------------------------------------------------------------------
%	DOCUMENT INFORMATION
%----------------------------------------------------------------------------------------

\title{ Universidad Sim\'on Bol\'ivar \\  MODELO CONCEPTUAL \\ BASE DE DATOS DE VENTRUEPE \\} % Title


\date{7 de Mayo de 2015} % Date for the report

\begin{document}

\maketitle % Insert the title, author and date

\begin{center}
\begin{tabular}{l r}
Fecha: & 7 de Mayo de 2015 \\ % Date the experiment was performed
Partners: & Ricardo M\"unch  \\ % Partner names
& Manuel Gomes \\
Profesores: & Leonid Tineo \\
& Darwin Rocha % Instructor/supervisor
\end{tabular}
\end{center}

\clearpage
% If you wish to include an abstract, uncomment the lines below
% \begin{abstract}
% Abstract text
% \end{abstract}

%----------------------------------------------------------------------------------------
%	SECTION 1
%----------------------------------------------------------------------------------------

\section{\'Indice}



\clearpage

\section{Introducci\'on}


Se pretende realizar un modelo conceptual de una base de datos propios del funcionamiento de la administración y servicios que le presta al cliente la mega-tienda de muebles VenTruePe. Para este objetivo utilizaremos el modelo de datos Entidad-Relaci\'on Extendido como base para el esquema que ilustrar\'a el resultado final. Adem\'as presentaremos un diccionario de datos que tendr\'a reflefado todas las entidades del esquema con su sem\'antica, dominnio, atributos y sem\'antica de los atributos adem\'as de las relaciones y su sem\'antica.
\\

Esperamos que el modelo conceptual resultante cumpla, en la medida de lo posible, con las condiciones de de correctitud, completitud, minimalidad, expresividad, legibilidad, extensibilidad y autoexplicatividad, que caracterizan un modelado conceptual de calidad.

   

\clearpage

\section{Contenido}

\subsection{Planteamiento del Problema}

Realizar un modelo conceptual que sirva a la mega-tienda de muebles VenTruePe, que le ayude a tener una mejor organizaci\'on y resolver los siguientes problemas:
\begin{itemize}
\item Organizar y llevar control riguroso de las compras que realiza la mega-tienda a sus proveedores. 
\item Controlar eficientemente el manejo de los recursos humanos de la empresa, todo lo relacionado los empleados, sus p\'olizas de seguro y las de sus dependientes.
\item Manejar un registro de las ventas, trueques y pedidos que realiza la mega-tienda.

\end{itemize}
 Como objetivo nos planteamos la creaci\'on de un esquema conceptual de bases de datos que satisfaga las necesidades de la empresa, que sea de calidad y que, adem\'as sea eficiente.
\subsection{Fundamentos te\'oricos}
Usaremos el modelo de datos Entidad-Relaci\'on extendido para resolver los problemas de la mega-tienda. Esta estructura nos servir\'a para el an\'alisis y la creaci\'on del esquema del modelo solicitado por la empresa.
\\

Se usar\'a, para la realizaci\'on del esquema, la herramienta de diagramaci\'on Dia.

% If you have more than one objective, uncomment the below:
%\begin{description}
%\item[First Objective] \hfill \\
%Objective 1 text
%\item[Second Objective] \hfill \\
%Objective 2 text
%\end{description}

\subsection{Soluci\'on del problema}

La soluci\'on const\'o de dos fases, la primera fue aquella relacionado con la adquisici\'on de conocimientos del problema, los requerimientos de la empresa y el funcionamiento de b\'asico de la mega-tienda, y la segunda se refiere a la realizaci\'on del diagrama y el dise\~no conceptual.

Los siguientes son aspectos que ayudar a comprender f\'acilmente el esquema que conecbimos como soluci\'on al problema pantleado de VenTruePe:


\begin{description}
\item[FACTURA\_P y FACTURA\_C:]
Se crearon dos entidades para modelar las diferentes facturas que interven\'ian en el problema de la mega-tienda, estas son las facturas que entrega la empresa a sus clientes por cada compra (FACTURA\_C) y las que recibe por parte de los proveedores de productos con relacionada con cada lote enviado por esta \'ultima entidad para mantener un inventario completo en todo momento (FACTURA\_P). \\

\item[Ap\'endice:]
Se presenta un ap\'endice que tendr\'a como contenido el diccionario de datos de nuestro modelo. Adem\'as explicaci\'on de las entidades d\'ebiles, especializaciones y categorizaciones.
\end{description} 
 
%----------------------------------------------------------------------------------------
%	SECTION 2
%----------------------------------------------------------------------------------------

\section{Conclusiones}

Luego de realizar este modelado conceptual de una base de datos que pretende poder resolver los problemas propuestos de la mega-tienda VenTruePe que cumpla con las condiciones de eficiencia y cualidad que un proyecto de este estilo merece. Esto se logr\'o luego de pasar por un proceso de conocimiento de los datos del problema, lo que se qur\'ia lograr y a partir de ah\'i, haciendo uso de los conceptos b\'asicos y estructuras de el modelo de datos Entidad-Relaci\'on Extendido, se realizaron varios bocetos, que fueron refin\'andose hasta lograr el resultado que presentamos. Logramos un nivel de abstracci\'on que nos ayudar\'a a realizar la base de datos que requiere la empresa de la mejor manera posible.
\clearpage
%----------------------------------------------------------------------------------------
%	SECTION 3
%----------------------------------------------------------------------------------------

\section{Bibliograf\'ia}

\begin{itemize}
\item L\'aminas del curso, Prof. Leonid TINEO, PhD.
\item “Fundamentals of Database Systems”. R. Elmasri y S. Navathe. 6th edition. Pearson/Addison-Wesley, 2011
\end{itemize}

\clearpage
\section{Ap\'endice}
\subsection{Diccionario de datos}

\begin{table}[h]
\resizebox{\textwidth}{!}{\begin{tabular}{|l|l|l|l|l|}
\hline
 \textbf{ENTIDAD}&\textbf{SEM\'ANTICA } & \textbf{ATRIBUTOS} & \textbf{SEM\'ANTICA } &  \textbf{DOMINIO} \\ 
 &\textbf{DE LA ENTIDAD} & & \textbf{DE LOS } & \\
 & & & \textbf{ATRIBUTOS} &\\ \hline
Persona & Toda clase de persona & Direcci\'on & Direcci\'on de  &  Secuencia \\ 
 & que participa en el modelo,  &  & habitaci\'on de la  & de caracteres. \\ 
 & & & persona. &\\ \cline{3-5} 
 & entidad a categorizar. &  &  & \\
 & & Edad& Edad de la & Entero.\\ 
 & & &persona. & \\ \cline{3-5}
 & & Sexo & Sexo de la& Cadena de   \\
 & & & persona. & caracteres.\\ \cline{3-5}
  & & Nombre & Nombre de la  & Cadena de \\
 & & & persona. & caracteres.\\ \cline{3-5}
 & & Apellido. & Apellido de la & Cadena de \\
 & & & persona.&caracteres.\\ \cline{3-5}
  & &Tel\'efono & Tel\'efonos de la & Secuencia de 11\\
 & & &persona. & caracteres\\ 
  & & & & num\'ericos.\\ \cline{3-5}
 & & Correo & Correo de la & Secuencia de \\
 & & & persona.& caracteres\\ \hline
Con\_C\'edula & Persona que  & C\'edula & C\'edula de la& Secuencia de\\
 & posee c\'edula de & &persona. & caracteres\\
 & &  & & num\'ericos \\ \hline
Dependiente & Persona que  & & &\\
 & depende de alg\'un& & &\\
 & empleado de la& & &\\
 & empresa. & & &\\ \hline
 Prima& Prima de una & Tipo & Especificaciones del  &\\
 & p\'oliza de &  & tipo de la prima &\\ \cline{3-5}
 & seguros para & Sexo & Sexo del & Secuencia de \\
 & un dependiente. & & dependiente. & caracteres.\\ \cline{3-5}
 & & Grupo\_etario  & Grupo etario & cadena de \\
 & & & al que pertenece el &caracteres.\\
 & & & dependiente.&\\ \cline{3-5}
 & & Monto & Monto a pagar & Cadena de \\
 & & & de la prima por &caracteres \\
 & & & dependiente. &num\'ericos.\\ \hline
Empleado & Persona que & Fecha\_inicio & Fecha de cuando & Secuencia de \\
 & trabaja para la & & empez\'o a trabajar& caracteres de la \\
 & empresa. & & para la empresa. & forma XX/XX/XX\\ \cline{3-5}
 & & Sueldo\_base & Sueldo base del & Secuencia de \\
 & & & empleado.& caracteres num\'ericos \\ \hline
Vendedor & Vendedores de & Categor\'ia & Categor\'ia de & Cadena de \\
 & la empresa. & & vendedor. & caracteres \\
 & & & & (I,II \'o III).\\ \cline{3-5}
 & &Num\_s\'abados & Cantidad de s\'abados & Caracter \\
 & & & que trabaja un & num\'erico\\
 & & & vendedor.&\\ \cline{3-5}
 & &Turno & Turno en el & Cadena de \\
 & & &que trabaja el  & caracteres.\\
 & & & empleado.&\\ \hline
 Administrativo&Persona que es & Cargo& Cargo del & Cadena de \\
 & parte del personal & & empleado administrativo. & caracteres\\
 & administrativo de& & &\\
 & la empresa.& & &\\ \hline
Cliente & Cliente de la& & &\\
 & mega-tienda& & &\\ \hline

\end{tabular}}
\end{table}









\begin{table}[h]
\resizebox{\textwidth}{!}{\begin{tabular}{|l|l|l|l|l|}
\hline

Operaci\'on & Opreci\'on a &Fecha & Fecha de & Cadena de \\
 & realizar por el & & realizaci\'on de & caracteres de\\
 & cliente en la& & la operaci\'on &la forma XX/XX/XX\\
 & mega-tienda. & & &\\ \hline
Trueque & Operaci\'on de  & & &\\
 & trueque.& & &\\ \hline
Venta & Operaci\'on de & & &\\
 & venta. & & &\\ \hline
Pedido & Operaci\'on de & & &\\
 & pedido de un  & & &\\
 & producto que no& & &\\
 & se encuentre en& & &\\
 & la mega-tienda.& & &\\ \hline
Recibo & Recibo que & Forma\_pago &Forma de pago & Cadena de\\
 & hace constar que& & del cliente &caracteres.\\ \cline{3-5}
 & se realiz\'o & num_tarjeta & Numero de la & Cadena de\\
 & un truque.& & tarjeta con la  & caracteres.\\
 & & & que se pag\'o. & num\'ericos \\ \cline{3-5}
 & & Clave_conf & Clave de  & Cadena de \\
 & & & confirmaci\'on & caracteres. \\ \cline{3-5}
 & & Serial & Serial del & Cadena de \\
 & & & recibo. & caracters.\\ \cline{3-5}
 & & Monto & Monto de la & Cadena de \\
 & & & transacci\'on & caracteres\\
 & & & & num\'ericos.\\ \hline
Factura\_c & Factura que & Forma_pago & Forma de pago & Cadena de \\
 & se le entrega& & del cliente. & caracteres.\\ \cline{3-5}
 & al cliente & Num_tarjeta & N\'umero de la & Cadena de \\
 & luego de una & & tarjeta usada & caracteres\\
 & venta & & para pagar. & num\'ericos\\ \cline{3-5}
 & & Clave\_conf & Clave de & Cadena de \\
 & & & confirmaci\'on de & caracteres.\\
 & & & la venta &\\ \cline{3-5}
 & & Serial & Serial de la & Cadena de \\
 & & & factura. & caracteres.\\ \cline{3-5}
Producto & Entidad abstracta & Descripci\'on & Descripci\'on del & Cadena de\\
 & de un producto.& & producto & caracteres\\ \cline{3-5}
 & & C\'odigo & C\'odigo del & Cadena de \\
 & & & producto. & caracteres \\ \cline{3-5}
 & & Nombre & Nombre del & cadena de \\
 & & & producto & caracteres \\ \hline
Art\'iculo & Instancia del & Categor\'ia & Categor\'ia del & Cadena de\\
 & producto. & & producto & caracteres.\\ \cline{3-5}
 & & Catidad & Cantidad de & Cadena de \\
 & & & un art\'iculo en & caracteres \\
 & & & existencia & num\'ericos. \\ \cline{3-5}
 & & Cantidad\_n & Cantidad en & Cadena de\\
 & & & existencia de & caracteres \\
 & & & art\'iculos nuevos.& num\'ericos.\\ \cline{3-5}
 & & Cantidad_u & Cantidad en & Cadena de\\
 & & & existencia de & caracteres \\
 & & & art\'iculos usados & num\'ericos. \\ \cline{3-5}
 & & Fecha_elab & Fecha de & Cadena de\\
 & & & elaboraci\'on del & caracteres num\'ericos \\
 & & & art\'iculo & de la forma XX/XX/XX.\\ \cline{3-5}
 & & Precio & Precio del & Cadena de \\
 & & & art\'iculo & caracteres n\'umericos. \\ \cline{3-5}
 & & & &\\
 & & & &\\
 & & & &\\
 & & & &\\
 & & & &\\
 & & & &\\
Usado & Clasificaci\'on de & Num\_inv & Cantidad en & Cadena de \\
 & art\'iculos usados. & & existencia de & caracteres num\'ericos\\
 & & & art\'iculos usados. & \\ \cline{3-5}
 & & & &\\
 & & & &\\
 & & & &\\
 & & & &\\
 & & & &\\
 & & & &\\
 & & & &\\
 & & & &\\
 & & & &\\
 & & & &\\
 & & & &\\
 & & & &\\
 & & & &\\
 & & & &\\
 & & & &\\
 & & & &\\
 & & & &\\
 & & & &\\
 & & & &\\
 & & & &\\
 & & & &\\
 & & & &\\
 & & & &\\
 & & & &\\
 & & & &\\
 & & & &\\
 & & & &\\
 & & & &\\
 & & & &\\
 & & & &\\
 & & & &\\
 & & & &\\
 & & & &\\
 & & & &\\
 & & & &\\
 & & & &\\
 & & & &\\
 & & & &\\
 & & & &\\
 
 
 \end{tabular}}
\end{table}
\end{document}
