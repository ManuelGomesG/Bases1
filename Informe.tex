%%%%%%%%%%%%%%%%%%%%%%%%%%%%%%%%%%%%%%%%%
% University/School Laboratory Report
% LaTeX Template
% Version 3.1 (25/3/14)
%
% This template has been downloaded from:
% http://www.LaTeXTemplates.com
%
% Original author:
% Linux and Unix Users Group at Virginia Tech Wiki 
% (https://vtluug.org/wiki/Example_LaTeX_chem_lab_report)
%
% License:
% CC BY-NC-SA 3.0 (http://creativecommons.org/licenses/by-nc-sa/3.0/)
%
%%%%%%%%%%%%%%%%%%%%%%%%%%%%%%%%%%%%%%%%%

%----------------------------------------------------------------------------------------
%	PACKAGES AND DOCUMENT CONFIGURATIONS
%----------------------------------------------------------------------------------------

\documentclass{article}

\usepackage[version=3]{mhchem} % Package for chemical equation typesetting
\usepackage{siunitx} % Provides the \SI{}{} and \si{} command for typesetting SI units
\usepackage{graphicx} % Required for the inclusion of images
\usepackage{natbib} % Required to change bibliography style to APA
\usepackage{amsmath} % Required for some math elements
\usepackage{makeidx} 

\setlength\parindent{0pt} % Removes all indentation from paragraphs

\renewcommand{\labelenumi}{\alph{enumi}.} % Make numbering in the enumerate environment by letter rather than number (e.g. section 6)

%\usepackage{times} % Uncomment to use the Times New Roman font

%----------------------------------------------------------------------------------------
%	DOCUMENT INFORMATION
%----------------------------------------------------------------------------------------

\title{MODELO CONCEPTUAL \\ BASE DE DATOS DE VENTRUEPE} % Title


\date{7 de Mayo de 2015} % Date for the report

\begin{document}

\maketitle % Insert the title, author and date

\begin{center}
\begin{tabular}{l r}
Fecha: & 7 de Mayo de 2015 \\ % Date the experiment was performed
Partners: & Ricardo M\"unch  \\ % Partner names
& Manuel Gomes \\
Profesores: & Leonid Tineo \\
Darwin Rocha % Instructor/supervisor
\end{tabular}
\end{center}

\clearpage
% If you wish to include an abstract, uncomment the lines below
% \begin{abstract}
% Abstract text
% \end{abstract}

%----------------------------------------------------------------------------------------
%	SECTION 1
%----------------------------------------------------------------------------------------

\section{\'Indice}



\clearpage

\section{Introducci\'on}


Se pretende realizar un modelo conceptual de una base de datos propios del funcionamiento de la administración y servicios que le presta al cliente la mega-tienda de muebles VenTruePe. Para este objetivo utilizaremos el modelo de datos Entidad-Relaci\'on Extendido como base para el esquema que ilustrar\'a el resultado final. Adem\'as presentaremos un diccionario de datos que tendr\'a reflefado todas las entidades del esquema con su sem\'antica, dominnio, atributos y sem\'antica de los atributos adem\'as de las relaciones y su sem\'antica.
\\

Esperamos que el modelo conceptual resultante cumpla, en la medida de lo posible, con las condiciones de de correctitud, completitud, minimalidad, expresividad, legibilidad, extensibilidad y autoexplicatividad, que caracterizan un modelado conceptual de calidad.

   

\clearpage

\section{Contenido}

\subsection{Planteamiento del Problema}

Realizar un modelo conceptual que sirva a la mega-tienda de muebles VenTruePe, que le ayude a tener una mejor organizaci\'on y resolver los siguientes problemas:
\begin{itemize}
\item Organizar y llevar control riguroso de las compras que realiza la mega-tienda a sus proveedores. 
\item Controlar eficientemente el manejo de los recursos humanos de la empresa, todo lo relacionado los empleados, sus p\'olizas de seguro y las de sus dependientes.
\item Manejar un registro de las ventas, trueques y pedidos que realiza la mega-tienda.

\end{itemize}
 Como objetivo nos planteamos la creaci\'on de un esquema conceptual de bases de datos que satisfaga las necesidades de la empresa, que sea de calidad y que, adem\'as sea eficiente.
\subsection{Fundamentos te\'oricos}
Usaremos el modelo de datos Entidad-Relaci\'on extendido para resolver los problemas de la mega-tienda. Esta estructura nos servir\'a para el an\'alisis y la creaci\'on del esquema del modelo solicitado por la empresa.
\\

Se usar\'a, para la realizaci\'on del esquema, la herramienta de diagramaci\'on Dia.

% If you have more than one objective, uncomment the below:
%\begin{description}
%\item[First Objective] \hfill \\
%Objective 1 text
%\item[Second Objective] \hfill \\
%Objective 2 text
%\end{description}

\subsection{Soluci\'on del problema}

La soluci\'on const\'o de dos fases, la primera fue aquella relacionado con la adquisici\'on de conocimientos del problema, los requerimientos de la empresa y el funcionamiento de b\'asico de la mega-tienda, y la segunda se refiere a la realizaci\'on del diagrama y el dise\~no conceptual.

El resultado de la  
\label{definitions}
\begin{description}
\item[Stoichiometry]
The relationship between the relative quantities of substances taking part in a reaction or forming a compound, typically a ratio of whole integers.
\item[Atomic mass]
The mass of an atom of a chemical element expressed in atomic mass units. It is approximately equivalent to the number of protons and neutrons in the atom (the mass number) or to the average number allowing for the relative abundances of different isotopes. 
\end{description} 
 
%----------------------------------------------------------------------------------------
%	SECTION 2
%----------------------------------------------------------------------------------------

\section{Experimental Data}

\begin{tabular}{ll}
Mass of empty crucible & \SI{7.28}{\gram}\\
Mass of crucible and magnesium before heating & \SI{8.59}{\gram}\\
Mass of crucible and magnesium oxide after heating & \SI{9.46}{\gram}\\
Balance used & \#4\\
Magnesium from sample bottle & \#1
\end{tabular}

%----------------------------------------------------------------------------------------
%	SECTION 3
%----------------------------------------------------------------------------------------

\section{Sample Calculation}

\begin{tabular}{ll}
Mass of magnesium metal & = \SI{8.59}{\gram} - \SI{7.28}{\gram}\\
& = \SI{1.31}{\gram}\\
Mass of magnesium oxide & = \SI{9.46}{\gram} - \SI{7.28}{\gram}\\
& = \SI{2.18}{\gram}\\
Mass of oxygen & = \SI{2.18}{\gram} - \SI{1.31}{\gram}\\
& = \SI{0.87}{\gram}
\end{tabular}

Because of this reaction, the required ratio is the atomic weight of magnesium: \SI{16.00}{\gram} of oxygen as experimental mass of Mg: experimental mass of oxygen or $\frac{x}{1.31}=\frac{16}{0.87}$ from which, $M_{\ce{Mg}} = 16.00 \times \frac{1.31}{0.87} = 24.1 = \SI{24}{\gram\per\mole}$ (to two significant figures).

%----------------------------------------------------------------------------------------
%	SECTION 4
%----------------------------------------------------------------------------------------

\section{Results and Conclusions}

The atomic weight of magnesium is concluded to be \SI{24}{\gram\per\mol}, as determined by the stoichiometry of its chemical combination with oxygen. This result is in agreement with the accepted value.

\begin{figure}[h]
\begin{center}

\caption{Figure caption.}
\end{center}
\end{figure}

%----------------------------------------------------------------------------------------
%	SECTION 5
%----------------------------------------------------------------------------------------

\section{Discussion of Experimental Uncertainty}

The accepted value (periodic table) is \SI{24.3}{\gram\per\mole} \cite{Smith:2012qr}. The percentage discrepancy between the accepted value and the result obtained here is 1.3\%. Because only a single measurement was made, it is not possible to calculate an estimated standard deviation.

The most obvious source of experimental uncertainty is the limited precision of the balance. Other potential sources of experimental uncertainty are: the reaction might not be complete; if not enough time was allowed for total oxidation, less than complete oxidation of the magnesium might have, in part, reacted with nitrogen in the air (incorrect reaction); the magnesium oxide might have absorbed water from the air, and thus weigh ``too much." Because the result obtained is close to the accepted value it is possible that some of these experimental uncertainties have fortuitously cancelled one another.

%----------------------------------------------------------------------------------------
%	SECTION 6
%----------------------------------------------------------------------------------------

\section{Answers to Definitions}

\begin{enumerate}
\begin{item}
The \emph{atomic weight of an element} is the relative weight of one of its atoms compared to C-12 with a weight of 12.0000000$\ldots$, hydrogen with a weight of 1.008, to oxygen with a weight of 16.00. Atomic weight is also the average weight of all the atoms of that element as they occur in nature.
\end{item}
\begin{item}
The \emph{units of atomic weight} are two-fold, with an identical numerical value. They are g/mole of atoms (or just g/mol) or amu/atom.
\end{item}
\begin{item}
\emph{Percentage discrepancy} between an accepted (literature) value and an experimental value is
\begin{equation*}
\frac{\mathrm{experimental\;result} - \mathrm{accepted\;result}}{\mathrm{accepted\;result}}
\end{equation*}
\end{item}
\end{enumerate}

%----------------------------------------------------------------------------------------
%	BIBLIOGRAPHY
%----------------------------------------------------------------------------------------

\bibliographystyle{apalike}

\bibliography{sample}

%----------------------------------------------------------------------------------------


\end{document}