%%%%%%%%%%%%%%%%%%%%%%%%%%%%%%%%%%%%%%%%%
% University/School Laboratory Report
% LaTeX Template
% Version 3.1 (25/3/14)
%
% This template has been downloaded from:
% http://www.LaTeXTemplates.com
%
% Original author:
% Linux and Unix Users Group at Virginia Tech Wiki 
% (https://vtluug.org/wiki/Example_LaTeX_chem_lab_report)
%
% License:
% CC BY-NC-SA 3.0 (http://creativecommons.org/licenses/by-nc-sa/3.0/)
%
%%%%%%%%%%%%%%%%%%%%%%%%%%%%%%%%%%%%%%%%%

%----------------------------------------------------------------------------------------
%	PACKAGES AND DOCUMENT CONFIGURATIONS
%----------------------------------------------------------------------------------------

\documentclass{article}

\usepackage[version=3]{mhchem} % Package for chemical equation typesetting
\usepackage{siunitx} % Provides the \SI{}{} and \si{} command for typesetting SI units
\usepackage{graphicx} % Required for the inclusion of images
\usepackage{natbib} % Required to change bibliography style to APA
\usepackage{amsmath} % Required for some math elements
\usepackage{makeidx} 

\setlength\parindent{0pt} % Removes all indentation from paragraphs

\renewcommand{\labelenumi}{\alph{enumi}.} % Make numbering in the enumerate environment by letter rather than number (e.g. section 6)

%\usepackage{times} % Uncomment to use the Times New Roman font

%----------------------------------------------------------------------------------------
%	DOCUMENT INFORMATION
%----------------------------------------------------------------------------------------

\title{ Universidad Sim\'on Bol\'ivar \\  MODELO CONCEPTUAL \\ BASE DE DATOS DE VENTRUEPE \\} % Title


\date{7 de Mayo de 2015} % Date for the report

\begin{document}

\maketitle % Insert the title, author and date

\begin{center}
\begin{tabular}{l r}
Fecha: & 7 de Mayo de 2015 \\ % Date the experiment was performed
Partners: & Ricardo M\"unch  \\ % Partner names
& Manuel Gomes \\
Profesores: & Leonid Tineo \\
& Darwin Rocha % Instructor/supervisor
\end{tabular}
\end{center}

\clearpage
% If you wish to include an abstract, uncomment the lines below
% \begin{abstract}
% Abstract text
% \end{abstract}

%----------------------------------------------------------------------------------------
%	SECTION 1
%----------------------------------------------------------------------------------------

\section{\'Indice}



\clearpage

\section{Introducci\'on}


Se pretende realizar un modelo conceptual de una base de datos propios del funcionamiento de la administración y servicios que le presta al cliente la mega-tienda de muebles VenTruePe. Para este objetivo utilizaremos el modelo de datos Entidad-Relaci\'on Extendido como base para el esquema que ilustrar\'a el resultado final. Adem\'as presentaremos un diccionario de datos que tendr\'a reflefado todas las entidades del esquema con su sem\'antica, dominnio, atributos y sem\'antica de los atributos adem\'as de las relaciones y su sem\'antica.
\\

Esperamos que el modelo conceptual resultante cumpla, en la medida de lo posible, con las condiciones de de correctitud, completitud, minimalidad, expresividad, legibilidad, extensibilidad y autoexplicatividad, que caracterizan un modelado conceptual de calidad.

   

\clearpage

\section{Contenido}

\subsection{Planteamiento del Problema}

Realizar un modelo conceptual que sirva a la mega-tienda de muebles VenTruePe, que le ayude a tener una mejor organizaci\'on y resolver los siguientes problemas:
\begin{itemize}
\item Organizar y llevar control riguroso de las compras que realiza la mega-tienda a sus proveedores. 
\item Controlar eficientemente el manejo de los recursos humanos de la empresa, todo lo relacionado los empleados, sus p\'olizas de seguro y las de sus dependientes.
\item Manejar un registro de las ventas, trueques y pedidos que realiza la mega-tienda.

\end{itemize}
 Como objetivo nos planteamos la creaci\'on de un esquema conceptual de bases de datos que satisfaga las necesidades de la empresa, que sea de calidad y que, adem\'as sea eficiente.
\subsection{Fundamentos te\'oricos}
Usaremos el modelo de datos Entidad-Relaci\'on extendido para resolver los problemas de la mega-tienda. Esta estructura nos servir\'a para el an\'alisis y la creaci\'on del esquema del modelo solicitado por la empresa.
\\

Se usar\'a, para la realizaci\'on del esquema, la herramienta de diagramaci\'on Dia.

% If you have more than one objective, uncomment the below:
%\begin{description}
%\item[First Objective] \hfill \\
%Objective 1 text
%\item[Second Objective] \hfill \\
%Objective 2 text
%\end{description}

\subsection{Soluci\'on del problema}

La soluci\'on const\'o de dos fases, la primera fue aquella relacionado con la adquisici\'on de conocimientos del problema, los requerimientos de la empresa y el funcionamiento de b\'asico de la mega-tienda, y la segunda se refiere a la realizaci\'on del diagrama y el dise\~no conceptual.

Los siguientes son aspectos que ayudar a comprender f\'acilmente el esquema que conecbimos como soluci\'on al problema pantleado de VenTruePe:


\begin{description}
\item[FACTURA\_P y FACTURA\_C:]
Se crearon dos entidades para modelar las diferentes facturas que interven\'ian en el problema de la mega-tienda, estas son las facturas que entrega la empresa a sus clientes por cada compra (FACTURA\_C) y las que recibe por parte de los proveedores de productos con relacionada con cada lote enviado por esta \'ultima entidad para mantener un inventario completo en todo momento (FACTURA\_P). \\

\item[Ap\'endice:]
Se presenta un ap\'endice que tendr\'a como contenido el diccionario de datos de nuestro modelo. Adem\'as explicaci\'on de las entidades d\'ebiles, especializaciones y categorizaciones.
\end{description} 
 
%----------------------------------------------------------------------------------------
%	SECTION 2
%----------------------------------------------------------------------------------------

\section{Conclusiones}

Luego de realizar este modelado conceptual de una base de datos que pretende poder resolver los problemas propuestos de la mega-tienda VenTruePe que cumpla con las condiciones de eficiencia y cualidad que un proyecto de este estilo merece. Esto se logr\'o luego de pasar por un proceso de conocimiento de los datos del problema, lo que se qur\'ia lograr y a partir de ah\'i, haciendo uso de los conceptos b\'asicos y estructuras de el modelo de datos Entidad-Relaci\'on Extendido, se realizaron varios bocetos, que fueron refin\'andose hasta lograr el resultado que presentamos. Logramos un nivel de abstracci\'on que nos ayudar\'a a realizar la base de datos que requiere la empresa de la mejor manera posible.
%----------------------------------------------------------------------------------------
%	SECTION 3
%----------------------------------------------------------------------------------------





\end{document}
